% ---
% RESUMOS
% ---

% RESUMO em português
\setlength{\absparsep}{18pt} % ajusta o espaçamento dos parágrafos do resumo
\begin{resumo}
    
    Neste projeto é proposto um método para a caracterização do impacto da Ciência da Computação nas demais áreas do conhecimento onde, a partir de currículos obtidos na Plataforma Lattes, são criadas redes de coautoria divididas em períodos, com pesquisadores sendo representados por nós e suas coautorias representadas por vértices. A relevância desse trabalho recai na possibilidade de explorar métricas topológicas medidas nas redes de coautoria para evidenciar o impacto que diferentes áreas do conhecimento recebem da Ciência da Computação, possibilitando que o método seja reproduzido em diferentes cenários.
    
    O método proposto para a identificação de coautorias realiza no máximo uma comparação entre qualquer par de publicações, produz o mesmo resultado independentemente da ordem de entrada e garante que o grafo de uma coautoria é sempre uma clique.

    \textbf{Palavras-chaves}: redes de coautoria. Ciência da Computação. Brasil.
\end{resumo}

% ABSTRACT in english
% \begin{resumo}[Abstract]
% \begin{otherlanguage*}{english}
%     This is the english abstract.

%     \vspace{\onelineskip}

%     \noindent 
%     \textbf{Keywords}: coauthorship networks. Computer Science. Brazil.
% \end{otherlanguage*}
% \end{resumo}