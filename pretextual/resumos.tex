% ---
% RESUMOS
% ---

% RESUMO em português
\setlength{\absparsep}{18pt} % ajusta o espaçamento dos parágrafos do resumo
\begin{resumo}
    O estudo bibliométrico das coautorias permite revelar características interessantes sobre as relações entre os pesquisadores, áreas do conhecimento e inclusive países na ciência. Neste projeto é proposto um método para a caracterização do impacto da Ciência da Computação nas demais áreas do conhecimento onde, a partir de currículos obtidos na Plataforma Lattes, são criadas redes de coautoria, com pesquisadores sendo representados por nós e suas coautorias representadas por arestas.
    
    Os resultados obtidos indicam que a Ciência da Computação exerce um papel importante na produção científica da Engenharia Elétrica, mas não existem evidências que apontam impacto em outras áreas (em termos de coautoria). Todavia, a relevância desse trabalho recai na possibilidade de explorar métricas topológicas que podem ser medidas nas redes de coautoria para evidenciar o impacto que diferentes áreas do conhecimento exercem entre si.
    
    O método proposto para a identificação de coautorias realiza no máximo uma comparação entre qualquer par de publicações, produz o mesmo resultado independentemente da ordem de entrada e garante que a rede de uma coautoria seja sempre um clique.
    
    \textbf{Palavras-chaves}: redes de coautoria. colaboração científica. Lattes.
\end{resumo}
