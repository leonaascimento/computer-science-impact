\chapter[Trabalhos correlatos]{Trabalhos correlatos}

Alguns trabalhos têm abordado redes de coautoria para compreender como pesquisadores dentro de sua área de conhecimento \cite{franceschet2011collaboration, santin2016collaboration}, grupos de pesquisa \cite{delgado2014analyzing} e instituições \cite{ioannidis2008measuring} interagem.

Métodos para avaliar a contribuição de pesquisadores dentro de uma rede de coautoria \cite{liu2005co, franceschet2011collaboration}, e caracterizar a interação de pesquisadores em diferentes regiões e estados brasileiros \cite{sidone2016ciencia} também já foram propostos.

\citeonline{newman2004coauthorship} mostrou que as redes de coautoria formam redes de mundo pequeno, onde o caminho mais próximo entre quaisquer dois pesquisadores não ultrapassam um número pequeno de vértices. Ainda, foi observado que, na maioria dos casos, o caminho mínimo passa pelo coautor mais conectado.

Currículos da Plataforma Lattes foram utilizados para construir redes brasileiras de coautoria que serviram para avaliar a produtividade e caracterizar o comportamento de pesquisadores brasileiros por períodos \cite{mena2014brazilian}, onde foi notado um número crescente de publicações científicas sendo feitas em coautoria. Esse crescimento também foi visto em redes de coautoria canadenses \cite{lariviere2006canadian}.

Sob a perspectiva da colaboração entre países foi possível notar um crescimento acelerado das redes de coautoria a partir do final do século XX \cite{glanzel2004analysing} que seguem aumentando \cite{lariviere2006canadian}.

Dentro da Ciência da Computação, foi possível notar que, mundialmente, pequenas listas de dois ou três autores são as mais comuns em publicações científicas, autores produtivos são os que possuem maior número de coautores, e a rede de coautoria também apresenta características de redes de mundo pequeno \cite{franceschet2011collaboration}.

A ferramenta scriptLattes \cite{mena2009scriptlattes} está preparada para analisar a produção científica de um grupo de pesquisadores cadastrados na Plataforma Lattes e também construir as redes de coautoria entre os pesquisadores do grupo. O sistema SUCUPIRA \cite{alves2011sucupira} permite extrair informações dos currículos Lattes para identificar e fornece uma breve análise sobre os dados.

É importante destacar que nenhum procedimento do scriptLattes nem do SUCUPIRA foi utilizado. Todos os procedimentos de coleta, normalização dos dados e construção das redes foram desenvolvidos, bem como sua análise, e ainda há muito para ser explorado.
