\chapter[Trabalhos correlatos]{Trabalhos correlatos}

Alguns trabalhos têm abordado redes de coautoria para compreender como pesquisadores dentro de sua área de conhecimento \cite{mena2014brazilian} \cite{franceschet2011collaboration} \cite{santin2016collaboration}, grupos de pesquisa \cite{delgado2014analyzing} e instituições \cite{ioannidis2008measuring} interagem.

Métodos para avaliar a contribuição de pesquisadores dentro de uma rede de coautoria \cite{franceschet2011collaboration} \cite{liu2005co}, e caracterizar a interação de pesquisadores em diferentes regiões e estados brasileiros \cite{sidone2016ciencia} já foram propostos.

Currículos da Plataforma Lattes foram utilizados para construir redes brasileiras de coautoria que serviram para avaliar a produtividade e caracterizar o comportamento de pesquisadores brasileiros em períodos \cite{mena2014brazilian}.

Dentro da Ciência da Computação, foi possível notar que, no caso do mundo, pequenas listas de dois ou três autores são as mais comuns em publicações científicas, autores produtivos são os que possuem maior número de coautores, e a rede de coautoria apresenta características de redes de mundo pequeno \cite{franceschet2011collaboration}.

A ferramenta scriptLattes \cite{mena2009scriptlattes} está preparada para analisar a produção científica de um grupo de pesquisadores cadastrados na Plataforma Lattes \cite{mena2013prospecccao} e serviu como ponto de partida para a construção das redes brasileiras de coautoria mencionadas.

Segundo \citeonline{mena2014brazilian}, a rede brasileira de coautoria possui natureza interdisciplinar, com um número crescente de pesquisadores e publicações científicas feitas em coautoria.