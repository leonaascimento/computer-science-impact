\chapter{Conjunto de dados}

Para mensurar o impacto que a Ciência da Computação exerce nas demais áreas do conhecimento, é necessário partir de um conjunto de dados que englobe publicações científicas de todas as áreas, de tal modo que todas possam estar representadas.

Mundialmente, os bancos de dados mais usados para estudos bibliométricos são o Scopus (Elsevier) e Web of Science (WoS, Clarivate Analytics). Segundo \citeonline{boyack2019creation}, isso se deve ao fato de cobrirem grande parte das áreas de conhecimento e incluir as referências citadas, facilitando a criação de redes de citações.

No cenário brasileiro, a Plataforma Lattes, um banco de dados que armazena o currículo de diversos pesquisadores, é uma fonte de dados interessante por ter se tornado um padrão nacional para a avaliação de atividades acadêmicas e profissionais, onde a maioria dos pesquisadores de todas as áreas do conhecimento estão registrados \cite{mena2014brazilian}.

Assim, contando com mais de um milhão de currículos de pesquisadores que incluem desde sua formação acadêmica até as publicações científicas mais recentes, passando por suas áreas de atuação e premiações, a Plataforma Lattes se mostra um banco de dados adequado para o estudo.

Como o custo computacional para processar milhões de currículos e dezenas de milhões de publicações científicas seria bastante alto, decidiu-se trabalhar apenas com os currículos de pesquisadores com nível de formação acadêmica acima de doutorado.

Este conjunto de currículos foi cedido pelo Grupo de Pesquisa em Cientometria do CMCC/UFABC no formato XML, que realizou a coleta dos currículos em janeiro de 2019.