\chapter{Conjunto de dados}

\section{Redes bibliométricas de larga escala}

\section{Plataforma Lattes}

\section{Coleta de dados}

\section{Análise dos metadados}

\section{Características do conjunto de dados}

Para caracterizar precisamente o impacto que a Ciência da Computação exerce nas demais áreas do conhecimento, é necessária uma rede que proporcione uma representação de toda a ciência, isto é, uma rede global. Redes desse tipo são mais robustas em altos níveis de granularidade \cite{boyack2019creation}, pois os mesmos resultados tendem a ser encontrados independentemente do método aplicado.

Existem muitos bancos de dados que podem ser usados para se construir uma rede global. Os mundialmente mais usados em estudos de bibliometria e cientometria são o Scopus e Web of Science (WoS). Segundo \citeonline{boyack2019creation}, a dominância desses bancos de dados se deve ao fato de cobrirem grande parte das áreas de conhecimento e incluir as referências citadas, facilitando a criação de redes de citações.

No cenário brasileiro, a Plataforma Lattes é uma fonte de dados interessante por armazenar os currículos de diversos pesquisadores e ter se tornado um padrão nacional para a avaliação de atividades acadêmicas e profissionais, onde a maioria dos pesquisadores de todas as áreas do conhecimento estão registrados \cite{mena2014brazilian}.

Assim, contando com mais de um milhão de currículos de pesquisadores que incluem desde sua formação acadêmica até as publicações científicas mais recentes, passando por suas áreas de atuação e premiações, a Plataforma Lattes se mostra um banco de dados adequado para o estudo.

Como o custo computacional para processar milhões de currículos e dezenas de milhões de publicações científicas seria bastante alto, decidiu-se trabalhar apenas com os currículos de pesquisadores com nível de formação acadêmica acima de doutorado.

Este conjunto de currículos foi cedido pelo Grupo de Pesquisa em Cientometria do CMCC/UFABC no formato XML, que realizou a coleta dos currículos em janeiro de 2019.
