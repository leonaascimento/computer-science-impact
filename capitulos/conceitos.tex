\chapter[Conceitos básicos]{Conceitos básicos}

A ciência é um sistema complexo composto por pesquisadores produzindo conhecimento científico e tecnológico \cite{boyack2019creation}. Pesquisadores escrevem artigos que são publicados e difundidos em periódicos ou conferências e tipicamente trabalham em instituições como universidades e empresas. Redes bibliométricas são redes onde os nós são objetos, como artigos, periódicos e livros, e as arestas são relações entre esses objetos, como citações e coautorias \cite{glanzel2003bibliometrics}.

Para construir as redes de coautoria que permitem caracterizar o impacto que a Ciência da Computação exerce nas demais áreas do conhecimento, é necessária uma fonte de dados confiável, que possua um número suficientemente grande de publicações científicas e inclua todas as áreas do conhecimento, proporcionando uma representação de toda a ciência. A Plataforma Lattes é um sistema reconhecido pela comunidade científica brasileira e já serviu como fonte de dados para diferentes trabalhos na área \cite{alves2011sucupira, balancieri2005analise, dias2013modelagem, mena2013prospecccao}.

Nesse sentido, serão apresentados os conceitos básicos mais importantes relacionados ao projeto de pesquisa nas próximas seções.

\section{Redes bibliométricas de larga escala}

\citeonline{boyack2019creation} mencionam que, antigamente, as redes bibliométricas eram restritas a centenas ou milhares de objetos devido à falta de acesso a dados e capacidade computacional. Nos dias de hoje, redes compostas por milhões ou dezenas de milhões de objetos já podem ser construídas e analisadas. Essas redes de larga escala possibilitaram análises que simplesmente não eram possíveis no passado \cite{newman2001scientific, shiffrin2004mapping}, análises que requerem o contexto de redes completas para proporcionar resultados mais precisos e robustos em altos níveis de granularidade \cite{boyack2019creation}.

Parte da popularidade das bases de dados Scopus e Web of Science (WoS) se deve ao fato delas serem abrangentes e cobrirem quase todas as áreas do conhecimento \cite{boyack2019creation}. Todavia, conseguir acesso irrestrito a essas bases de dados depende de condições que nem todo pesquisador consegue satisfazer.

\section{Redes de coautoria científica}

Uma rede de coautoria científica é um tipo de rede social onde os nós são autores e as arestas a coautoria entre eles \cite{delgado2014analyzing, franceschet2011collaboration}, ou seja, indicam a existência de pelo menos uma publicação feita em conjunto.

Coautorias normalmente indicam colaboração mútua \cite{glanzel2003bibliometrics} e se traduzem em arestas não direcionadas, uma vez que é difícil decidir o que foi de fato foi realizado por cada um dos autores da publicação \cite{ioannidis2008measuring}.

\section{Plataforma Lattes}

A Plataforma Lattes é composta por um conjunto de bancos de dados e ferramentas construídas para auxiliar atividades de planejamento e gestão da ciência no Brasil. Armazenando informações sobre estudantes e pesquisadores do país, além de instituições e grupos de pesquisa em atividade, a Plataforma conta com uma grande variedade de fatos e informações sobre pesquisa e desenvolvimento \cite{alves2011sucupira, medeiros2013dynamics}.

O Currículo Lattes, um dos bancos de dados da Plataforma, foi um meio encontrado para padronizar o histórico das atividades profissionais, científicas e acadêmicas dos pesquisadores \cite{dias2013modelagem}. A maioria dos pesquisadores brasileiros de todas as áreas do conhecimento possui seu currículo registrado na Plataforma \cite{mena2014brazilian} e o atualiza como parte da sua rotina. Atualmente, grande parte das instituições acadêmicas usam esses dados para elaborar relatórios acadêmicos de produtividade e pesquisa \cite{mena2009scriptlattes}.

Assim, por ser reconhecido pela comunidade acadêmica brasileira, ter se tornado um padrão nacional para a avaliação de atividades acadêmicas e profissionais, conter um grande número de currículos de cadastrados, os quais se mantêm atualizados e por isso incluem desde a formação acadêmica até as produções acadêmicas e profissionais mais recentes dos pesquisadores, o Currículo Lattes, que atualmente consta com mais de seis milhões de currículos, se mostra um banco de dados suficientemente abrangente para o estudo da ciência brasileira.

% Em geral a escrita está bonita!!!
