\chapter{Considerações finais}

Foram construídas as redes de coautoria dos pesquisadores com nível igual ou acima de doutorado cadastrados na Plataforma Lattes. O objetivo era caracterizar o impacto que a Ciência da Computação exerce nas demais áreas do conhecimento para responder quais são as áreas que mais interagem com a Ciência da Computação, mensurar o impacto nas demais áreas do conhecimento, bem como o comportamento dessa medida ao longo do tempo.

Os resultados obtidos apontam que a Ciência da Computação é uma das cinco áreas que mais produzem publicações científicas em coautoria e que esse número vem crescendo consistentemente ao longo dos anos. Todavia, com relação à participação em grupos interdisciplinares para a produção de publicações científicas, a Ciência da Computação não está bem posicionada. Áreas como a Química e a Agronomia, que apresentam desempenho similar na produção de publicações em coautoria, também estão entre as primeiras colocadas neste quesito.

Apesar de ser uma área que se conecta a quase todas as áreas do conhecimento, as coautorias realizadas por cientistas da computação não exercem impacto significativo na produção das publicações científicas de outras áreas. Assim, mesmo observando que o uso da computação vem sendo cada vez mais explorado, isso não se traduz em um pesquisador da área participando do processo, o que abre espaço para novas discussões e sugere que pesquisadores de outras áreas estão se capacitando para o uso da computação como ferramenta.

\section{Limitações}

A base de dados é um retrato de janeiro de 2019 e, como a obtenção dos currículos é demorada, não é possível que o processo de descoberta do conhecimento considere sempre a versão mais atualizada dos dados. Como a identificação das coautorias dura aproximadamente 12 dias em uma máquina com dois núcleos de processamento, mesmo que um novo retrato da base de dados pudesse ser obtido a cada hora, não seria possível trabalhar com a versão mais recente do conjunto de dados.

O tempo de processamento é alto porque o algoritmo usado para a identificação das coautorias, mesmo com as otimizações, continua sendo $O(n^2)$ para uma lista de publicações de tamanho $n$. Ou seja, o tempo de execução aumenta quadraticamente em relação ao tamanho da entrada.

As análises feitas abordam o retrato atual das coautorias, mas não são capazes de informar se existiram mudanças nas relações de coautoria entre as áreas e como esse comportamento se deu ao longo dos anos. Ainda, nem todos os pesquisadores preenchem suas áreas e isso faz com que tenhamos que desconsiderar parte dos dados. Inclusive existem coautorias que foram desconsideradas devido a esse tipo de dado faltante.

\section{Trabalhos futuros}

Existem muitas análises que podem ser derivadas das redes de coautoria construídas, uma vez que elas dão o panorama geral das coautorias de todos os pesquisadores com nível de instrução igual ou acima de doutorado, os quais são os mais relevantes na produção científica brasileira.

Os dados presentes no Currículo Lattes permitem entender como funcionam as coautorias entre pesquisadores de universidades e empresas, de diferentes regiões do país e do mundo, além do impacto que falar uma língua estrangeira pode ter no aumento das coautorias

Também poderiam ser exploradas propriedades dessas redes, como a identificação de pesquisadores produtivos e influentes, além de observar a hipótese de que os autores mais produtivos são os que mais colaboram.

Momentos históricos do país poderiam ser avaliados contra o volume de publicações feitas em coautoria. A \autoref{fig:coautoriaanual} apresenta de 2012 para 2013 uma desaceleração no crescimento de publicações feitas em coautoria, o que coincide com a greve das universidades federais. Seria relevante descobrir se manifestações sociais e políticas, assim como surtos de doenças, influenciam as relações da ciência.

Ainda, poderia ser construída uma ferramenta para desambiguação autores, dado que identificar todos os coautores de uma publicação em citações é problema. Esse problema é ainda pior quando apenas o nome de um dos pesquisadores é colocado nas citações.
