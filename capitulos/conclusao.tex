%\chapter[Conclusão]{Conclusão}
\chapter{Considerações finais}

A ciência da computação é uma área que se conecta a quase todas as áreas do conhecimento que apareceram em nossa listagem, porém não exerce impacto significativo em suas produções bibliográficas. Isso contradiz a sensação de que a computação está envolvida em tudo nos dias de hoje.

Poderíamos dizer que os produtos da ciência da computação são bastante usados em nosso dia a dia e com certeza também nas pesquisas de outras áreas, porém profissionais da área não são requisitados para os estudos. Sendo assim, poderíamos apenas dizer que talvez os produtos da ciência da computação permeiem tudo e tenham impacto na pesquisa, mas não podemos dizer que a pesquisa realizada no brasil envolve pesquisadores da área.

Análises de caso de pesquisas que aconteceram poderiam ser feitas para entender qual o papel exercido pela ciência da computação nessas outras áreas e entender se é apenas o volume que é baixo.

\section{Limitações}

A base de dados é um retrato do mês de janeiro de 2019, e não ter os dados atualizados de forma mais ágil é um fator que dificulta a descoberta do conhecimento, acompanhando a mudança que acontece dia após dia.

Ainda, o processamento durou cerca de duas 1 semana e meia, o que significa dizer que se um novo retrato fosse feito hoje não conseguiriamos disponibilizar em tempo ábil.

O algoritmo usado para comparação dos título é $O(n^2)$ o que significa dizer que ele é quadrático em relação ao número de entrada. Os filtros aplicados de ano e similaridade por diferença de tamanho da string ajudam a diminuir o tempo de execução, mas não é suficiente.

Nem todos os pesquisadores preenchem suas áreas e isso faz com que tenhamos que desconsiderar boa parte dos dados. Assim, temos dados desconhecidos que não sabemos inferir sua área de atuação. Apesar de termos informações sobre a formação do pesquisador, não extraímos esse dado do currículo e portanto nao conseguimos usá-lo.

As análises estão pouco aprofundadas porque muito tempo foi investido no processamento dos dados e outras estratégias que foram aplicadas no passado fizeram o tempo ser mal investido. Houveram problemas com servidores. Mudança de ferramentas, gastos com computação em nuvem para auxiliar no processamento.

Houve a tentativa de realizar um procedimparalelizar os dados ento para 

\section{Trabalhos futuros}

Existem muitas análises que podem ser derivadas do banco de dados construído, uma vez que ele dá o panorama geral das coautorias de todos os pesquisadores com nivel de instruição de doutorado na plataforma lattes. Com os dados já presentes no banco de dados é possível entender como funcionam as coautorias entre as universidades, possibilitando entender qual é o nível de interação entre elas. Também é possível entender como anda a produção de materiais de divulgação científica, dado que esse indicador também foi coletado.

Também poderiam ser considerados estudos sobre a rede de coautoria, explorando suas propriedades, identificando os pesquisadores mais produtivos e mais influentes na comunidade científica brasileira. Foi observado por mena-chalco que os autores mais produtivos são os que mais colaboram, e um estudo dos assuntos abordados nessas colaborações também poderia ser feito aplicando técnicas de processamento de linguagem natural no título das publicações.

Pode ser construída uma ferramenta capaz de desambiguar autores. É um problema identificar todos os coautores de uma publicação em citações, pois muitas delas apresentam apenas o nome do primeiro pesquisador com o et. al.. A partir apenas do título da publicação e o ano poderíamos fornecer a lista completa de pelo menos os pesquisadores presentes na plataforma lattes.