\chapter{Resultados e Discussão}
Nesta seção apresentamos ...

\section{Produção bibliográfica}

A produção bibliográfica da Ciência da Computação está posicionada entre as 10 primeiras.

Se olharmos pelo tipo de publicação com maior volume vemos que o número de publicações em conferências é alto. Ainda, esse é um comportamento mais presente na Engenharia elétrica e etc.

Podemos notar um aumento no número de publicações bibliográficas, mas a taxa de crescimento segue o que é visto nas demais áreas, e portanto não notamos grandes mudanças após data X, quando a ciencia da computação...

\section{Colaboração}

A colaboração da ciencia da computação é expressiva apenas em algumas áreas. Ela está ligada à maioria das áreas, porém esse é um comportamento comum pelo que se pôde observar.

Ao longo do tempo podemos notar

\section{Interdisciplinaridade}

Interdisciplinaridade é um conceito que pode ser interpretado de diferentes maneiras. Vemos que a ciencia da computação participa com autores das mais variadas áreas, porém pela pouco impacto sofrido de outras áreas e pouco impacto exercido em outras áreas, com exceção da engenharia elétrica, engenharia de biomedicina e tal, não podemos afirmar que esta é uma área interdisciplinar.

Diferente da educação, que tem uma grande participação em boa parte das áreas. Ainda, vale notar que este é um resultado esperado e que pode servir como um indicador de que o método aplicado nos dados é válido e produz resultados que correspondem à realidade.

