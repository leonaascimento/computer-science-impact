\documentclass[
	% -- opções da classe memoir --
	12pt,                   % tamanho da fonte
	openany,                % capítulos começam em sequência
	twoside,                % para impressão em verso e anverso. Oposto a oneside
	a4paper,                % tamanho do papel. 
	% -- opções da classe abntex2 --
	%chapter=TITLE,         % títulos de capítulos convertidos em letras maiúsculas
	%section=TITLE,         % títulos de seções convertidos em letras maiúsculas
	%subsection=TITLE,      % títulos de subseções convertidos em letras maiúsculas
	%subsubsection=TITLE,   % títulos de subsubseções convertidos em letras maiúsculas
	% -- opções do pacote babel --
	english,                % idioma adicional para hifenização
	%french,                % idioma adicional para hifenização
	%spanish,               % idioma adicional para hifenização
	brazil                  % o último idioma é o principal do documento
	]{abntex2}

% ---------------------
% Pacotes OBRIGATÓRIOS
% ---------------------
\usepackage{lmodern}            % Usa a fonte Latin Modern			
\usepackage[T1]{fontenc}        % Selecao de codigos de fonte.
\usepackage[utf8]{inputenc}     % Codificacao do documento (conversão automática dos acentos)
\usepackage{lastpage}           % Usado pela Ficha catalográfica
\usepackage{indentfirst}        % Indenta o primeiro parágrafo de cada seção.
\usepackage{color}              % Controle das cores
\usepackage{graphicx,graphicx}  % Inclusão de gráficos
\usepackage{epsfig,subfig}      % Inclusão de figuras
\usepackage{microtype}          % Melhorias de justificação
\usepackage{csvsimple}          % Importar arquivos csv como tabelas
\usepackage{lscape}             % Permitir rotacionar folha específica
\usepackage{rotating}           % Permitir rotacionar folha específica
\usepackage{pdflscape}          % Permitir rotacionar folha específica
% ---------------------
		
% ---------------------
% Pacotes ADICIONAIS
% ---------------------
\usepackage{amsmath,amssymb,mathrsfs}  % Comandos matemáticos avançados 
\usepackage{algpseudocode,algorithm}   % Para poder adicionar algoritmos
\usepackage{setspace}                  % Para permitir espaçamento simples, 1 1/2 e duplo
\usepackage{verbatim}                  % Para poder usar o ambiente "comment"
\usepackage{tabularx}                  % Para poder ter tabelas com colunas de largura auto-ajustável
\usepackage{afterpage}                 % Para executar um comando depois do fim da página corrente
\usepackage{url}                       % Para formatar URLs (endereços da Web)
% ---------------------

% ---------------------
% Pacotes de CITAÇÕES
% ---------------------
\usepackage[brazilian,hyperpageref]{backref} % Paginas com as citações na bibl
\usepackage[alf]{abntex2cite}                % Citações padrão ABNT (alfa)
%\usepackage[num]{abntex2cite}               % Citações padrão ABNT (numericas)
% ---------------------

% Configura fonte serifada para títulos
\renewcommand{\ABNTEXchapterfont}{\rmfamily}

% Configura citação de algoritmo
\makeatletter
\renewcommand{\ALG@name}{Algoritmo}
\makeatother
\newcommand{\algorithmautorefname}{Algoritmo}

% Permite chamar um método dentro de outro método em algoritos
\MakeRobust{\Call}

% Configurações de CITAÇÕES para abntex2
\include{extras/conf_citacoes}

% Inclusão de dados para CAPA e FOLHA DE ROSTO (título, autor, orientador, etc.)
% ---
% Informações de dados para CAPA e FOLHA DE ROSTO
% ---
\titulo{O impacto da Ciência da Computação nas diferentes áreas do conhecimento a partir da análise de redes brasileiras de coautoria}
\autor{Leonardo Nascimento}
\local{Santo André - SP}
\data{Dezembro de 2019}
\orientador{Prof. Dr. Jesús Pascual Mena-Chalco}
\instituicao{%
  Universidade Federal do ABC
  \par
  Centro de Matemática, Computação e Cognição (CMCC)
  \par
  Graduação em Ciência da Computação}
\tipotrabalho{Projeto de Graduação em Computação}
% O preambulo deve conter o tipo do trabalho, o objetivo,
% o nome da instituição e a área de concentração
\preambulo{\textbf{Projeto de Graduação em Computação} apresentado ao Centro de Matemática, Computação e Cognição (CMCC), como parte dos requisitos necessários para a obtenção do título de Bacharel em Ciência da Computação.}
% ---


% Inclui Configurações de aparência do PDF Final
\include{extras/conf_pdf}

% O tamanho da identação do parágrafo é dado por:
\setlength{\parindent}{1.3cm}

% Controle do espaçamento entre um parágrafo e outro:
\setlength{\parskip}{0.2cm}  % tente também \onelineskip

% ---------------------
% Compila o indice
% ---------------------
\makeindex
% ---------------------


%%%%%%%%%%%%%%%%%%%%%%%%%%%
%%  INICIO DO DOCUMENTO  %%
%%%%%%%%%%%%%%%%%%%%%%%%%%%
\begin{document}

% Retira espaço extra obsoleto entre as frases.
\frenchspacing

% ----------------------------------------------------------
% ELEMENTOS PRÉ-TEXTUAIS (Capa, Resumo, Abstract, etc.)
% ----------------------------------------------------------
\pretextual

% Capa
% ---
% Impressão da Capa
% ---
  \begin{capa}%
    \center
	\ABNTEXchapterfont\large{Universidade Federal do ABC \\ Centro de Engenharia, Modelagem e Ciências Sociais Aplicadas \\ Programa de Pós-Graduação em Engenharia da Informação}
	%\vspace{1.5cm}

    \vfill
    \ABNTEXchapterfont\bfseries\LARGE\imprimirtitulo
    \vfill

	%\vfill
	\ABNTEXchapterfont\large\imprimirautor
	\vfill
%
	
    \large\imprimirlocal \\ \large\imprimirdata

    \vspace*{1cm}
  \end{capa}
% ---

% Folha de rosto (o * indica que haverá a ficha bibliográfica)
\imprimirfolhaderosto*

% Imprimir Ficha Catalografica
% % ---
% Ficha Catalográfica
% ---
% Isto é um exemplo de Ficha Catalográfica, ou ``Dados internacionais de
% catalogação-na-publicação''. Você pode utilizar este modelo como referência. 
% Porém, talvez a biblioteca lhe fornece um PDF
% com a ficha catalográfica definitiva após a defesa do trabalho. Quando estiver
% com o documento, salve-o como PDF no diretório do seu projeto e substitua todo
% o conteúdo de implementação deste arquivo pelo comando abaixo:
%
% \begin{fichacatalografica}
%     \includepdf{fig_ficha_catalografica.pdf}
% \end{fichacatalografica}
\begin{fichacatalografica}
	\vspace*{\fill}					% Posição vertical
	\hrule							% Linha horizontal
	\begin{center}					% Minipage Centralizado
	\begin{minipage}[c]{12.5cm}		% Largura
	
	\imprimirautor
	
	\hspace{0.5cm} \imprimirtitulo  / \imprimirautor. --
	\imprimirlocal, \imprimirdata-
	
	\hspace{0.5cm} \pageref{LastPage} p. : il. (algumas color.) ; 30 cm.\\
	
	\hspace{0.5cm} \imprimirorientadorRotulo~\imprimirorientador\\
	
	\hspace{0.5cm}
	\parbox[t]{\textwidth}{\imprimirtipotrabalho~--~\imprimirinstituicao,
	\imprimirdata.}\\
	
	\hspace{0.5cm}
		1. Redes de coautoria.
		2. Ciência da computação.
		3. Brasil.
		I. Jesús Pascual Mena-Chalco.
		II. Universidade Federal do ABC.
		III. Centro de Matemática, Computação e Cognição.
		IV. O impacto da Ciência da Computação nas diferentes áreas do conhecimento a partir da análise de redes brasileiras de coautoria\\ 			
	
	\hspace{8.75cm} CDU 02:141:005.7\\
	
	\end{minipage}
	\end{center}
	\hrule
\end{fichacatalografica}
% ---

% Inserir Folha de Aprovação
\include{pretextual/assinaturas}

% Dedicatória
% ---
% Dedicatória
% ---
\begin{dedicatoria}
   \vspace*{\fill}
   \centering
   \noindent
   \textit{ À comunidade científica brasileira. } \vspace*{\fill}
\end{dedicatoria}
% ---

% Agradecimentos
% ---
% Agradecimentos
% ---
\begin{agradecimentos}

Agradeço aos meus pais, que me deram apoio e incentivo. Sou grato também aos meus familiares, que compreenderam minha ausência pelo tempo dedicado aos estudos. Obrigado à minha namorada, que não me deixou ser vencido pelo cansaço e me estimulou durante todos os anos da graduação.

\end{agradecimentos}
%% ---

% Epígrafe
% ---
% Epígrafe
% ---
\begin{epigrafe}
    \vspace*{\fill}
    \begin{flushright}
        \textit{``Não se conhece completamente uma ciência enquanto não se souber da sua história.'' \\ Auguste Comte}
    \end{flushright}
\end{epigrafe}
% ---

% Resumo e Abstract
% ---
% RESUMOS
% ---

% RESUMO em português
\setlength{\absparsep}{18pt} % ajusta o espaçamento dos parágrafos do resumo
\begin{resumo}
    O estudo bibliométrico das coautorias permite revelar características interessantes sobre as relações entre os pesquisadores, áreas do conhecimento e inclusive países na ciência. Neste projeto é proposto um método para a caracterização do impacto da Ciência da Computação nas demais áreas do conhecimento onde, a partir de currículos obtidos na Plataforma Lattes, são criadas redes de coautoria, com pesquisadores sendo representados por nós e suas coautorias representadas por arestas.
    
    Os resultados obtidos indicam que a Ciência da Computação exerce um papel importante na produção científica da Engenharia Elétrica, mas não existem evidências que apontam impacto em outras áreas (em termos de coautoria). Todavia, a relevância desse trabalho recai na possibilidade de explorar métricas topológicas que podem ser medidas nas redes de coautoria para evidenciar o impacto que diferentes áreas do conhecimento exercem entre si.
    
    O método proposto para a identificação de coautorias realiza no máximo uma comparação entre qualquer par de publicações, produz o mesmo resultado independentemente da ordem de entrada e garante que a rede de uma coautoria seja sempre um clique.
    
    \textbf{Palavras-chaves}: redes de coautoria. Ciência da Computação. Brasil.
\end{resumo}


% Lista de ilustrações
\pdfbookmark[0]{\listfigurename}{lof}
\listoffigures*
\cleardoublepage

% Lista de tabelas
\pdfbookmark[0]{\listtablename}{lot}
\listoftables*
\cleardoublepage

% Lista de algoritmos
\renewcommand{\listalgorithmname}{Lista de algoritmos}
\pdfbookmark[0]{\listalgorithmname}{loa}
\listofalgorithms
\cleardoublepage

% Lista de abreviaturas e siglas
% \begin{siglas}
%   \item[ABNT] Associação Brasileira de Normas Técnicas
%   \item[abnTeX] Normas para TeX
% \end{siglas}

% Lista de símbolos
% \begin{simbolos}
%   \item[$ \Gamma $] Letra grega Gama
%   \item[$ \Lambda $] Lambda
%   \item[$ \zeta $] Letra grega minúscula zeta
%   \item[$ \in $] Pertence
% \end{simbolos}

% Inserir o SUMÁRIO
\pdfbookmark[0]{\contentsname}{toc}
\tableofcontents*
\cleardoublepage

% ----------------------------------------------------------
% ELEMENTOS TEXTUAIS (Capítulos)
% ----------------------------------------------------------
\textual
% Elementos textuais com numeração arábica
\pagenumbering{arabic}
% Reinicia a contagem do número de páginas
\setcounter{page}{1}

% Inclui cada capitulo da Dissertação
% ----------------------------------------------------------
% Introdução 
% Capítulo sem numeração, mas presente no Sumário
% ----------------------------------------------------------

\chapter[Introdução]{Introdução}
\addcontentsline{toc}{chapter}{Introdução}

O número de pesquisas feitas em coautoria vêm aumentando no Brasil \cite{mena2014brazilian} e no mundo \cite{glanzel2003bibliometrics} em diferentes áreas do conhecimento, tanto entre áreas distintas quanto dentro de uma mesma área. Compreender a dinâmica dessas coautorias e medir o impacto \footnote{O termo impacto é subjetivo e, portanto, pode ter diversos significados \cite{roemer2015meaningful}. Todavia estes significados costumam englobar a ideia de um efeito com alguma intensidade \cite{roemer2015meaningful}. Neste trabalho, o termo impacto descreve interações ao longo do tempo entre as áreas do conhecimento encontradas em uma rede de coautoria.} que uma área do conhecimento pode ter em outras revolucionaria a forma que recursos para pesquisa são distribuídos.

A partir do estudo bibliográfico de publicações científicas podem ser formadas redes de coautoria, onde os nós são pesquisadores e os vértices a coautoria entre eles. Os pesquisadores mais influentes e as áreas de conhecimento mais populares, por exemplo, podem ser obtidos a partir de métricas topológicas calculadas nessas redes \cite{franceschet2011collaboration}.

Este estudo considera a Plataforma Lattes como fonte de dados para a formação das redes de coautoria, divididas em períodos, que serão objeto de estudo. A Plataforma armazena o currículo acadêmico de centenas de milhares de pesquisadores brasileiros, incluindo registros de atividades acadêmicas que vão desde a formação do pesquisador até suas publicações científicas mais recentes. Entende-se que esta fonte de dados, além de ser amplamente aceita e utilizada pela comunidade científica brasileira, se mantém atualizada.

As métricas topológicas dessas redes de coautoria serão calculadas e a análise do seu comportamento ao longo do tempo será feita, destacando o impacto da Ciência da Computação nas demais áreas do conhecimento.

Apesar de centrar nas áreas de conhecimento da comunidade científica brasileira, o método aplicado é geral e poderá ser reproduzido em outras comunidades científicas.

\section{Trabalhos correlatos}
\addcontentsline{toc}{section}{Trabalhos correlatos}

Alguns trabalhos têm abordado redes de coautoria para compreender como pesquisadores dentro de sua área de conhecimento \cite{mena2014brazilian} \cite{franceschet2011collaboration} \cite{santin2016collaboration}, grupos de pesquisa \cite{delgado2014analyzing} e instituições \cite{ioannidis2008measuring} interagem.

Métodos para avaliar a contribuição de pesquisadores dentro de uma rede de coautoria \cite{franceschet2011collaboration} \cite{liu2005co}, e caracterizar a interação de pesquisadores em diferentes regiões e estados brasileiros \cite{sidone2016ciencia} já foram propostos.

Currículos da Plataforma Lattes foram utilizados para construir redes brasileiras de coautoria que serviram para avaliar a produtividade e caracterizar o comportamento de pesquisadores brasileiros em períodos \cite{mena2014brazilian}.

Dentro da Ciência da Computação, foi possível notar que, no caso do mundo, pequenas listas de dois ou três autores são as mais comuns em publicações científicas, autores produtivos são os que possuem maior número de coautores, e a rede de coautoria apresenta características de redes de mundo pequeno \cite{franceschet2011collaboration}.

A ferramenta scriptLattes \cite{mena2009scriptlattes} está preparada para analisar a produção científica de um grupo de pesquisadores cadastrados na Plataforma Lattes \cite{mena2013prospecccao} e serviu como ponto de partida para a construção das redes brasileiras de coautoria mencionadas.

Segundo \citeonline{mena2014brazilian}, a rede brasileira de coautoria possui natureza interdisciplinar, com um número crescente de pesquisadores e publicações científicas feitas em coautoria.

\section{Objetivo}
\addcontentsline{toc}{section}{Objetivo}

Caracterizar o impacto da Ciência da Computação nas demais áreas do conhecimento a partir de indicadores \footnote{Neste trabalho, indicadores são métricas de redes complexas.} em redes de coautoria inter- e intra-áreas obtidos a partir de publicações científicas registradas em currículos da Plataforma Lattes, considerando os mais de 130 \footnote{O número total de pesquisadores formados no nível de doutorado, em 2016, é de 130.140. Obtido em http://lattes.cnpq.br/web/dgp/por-titulacao, último acesso em 31 de outubro de 2017.} mil pesquisadores com nível de instrução de Doutorado.

\subsection{Objetivos específicos}
\addcontentsline{toc}{subsection}{Objetivos específicos}

\begin{itemize}
\item Desenvolver um método capaz de identificar com alto grau de acurácia a grande área de atuação de um pesquisador com base nas informações de seu currículo Lattes;
\item Desenvolver um método capaz de identificar com alto grau de acurácia os coautores de uma publicação referenciando-os univocamente a um pesquisador contido no conjunto de pesquisadores estudado.
\end{itemize}

\subsection{Metas}
\addcontentsline{toc}{subsection}{Metas}
Responder quais áreas do conhecimento interagem com a Ciência da Computação, mensurar o impacto exercido entre a Ciência da Computação e as demais áreas do conhecimento, bem como o comportamento dessa medida ao longo do tempo.

\chapter[Conceitos básicos]{Conceitos básicos}

A ciência é um sistema complexo composto por pesquisadores produzindo conhecimento científico e tecnológico \cite{boyack2019creation}. Pesquisadores escrevem artigos que são publicados e difundidos em periódicos ou conferências e tipicamente trabalham em instituições como universidades e empresas. Redes bibliométricas são redes onde os nós são objetos, como artigos, periódicos e livros, e as arestas são relações entre esses objetos, como citações e coautorias.

Para caracterizar o impacto que a Ciência da Computação exerce nas demais áreas do conhecimento, é necessária uma fonte de dados confiável que possua um número suficientemente grande de publicações científicas e inclua todas as áreas do conhecimento, proporcionando uma representação de toda a ciência.

\textcolor{magenta}{
Nesse contexto, nesta seção apresentamos uma descrição dos conceitos básicos mais importantes relacionados com o projeto de pesquisa}


\section{Redes bibliométricas de larga escala}

\citeonline{boyack2019creation} mencionam que, antigamente, as redes bibliométricas eram restritas a centenas ou milhares de objetos devido à falta de acesso a dados e capacidade computacional. Nos dias de hoje, redes compostas por milhões ou dezenas de milhões de objetos já podem ser construídas e analisadas. Essas redes de larga escala possibilitaram análises que simplesmente não eram possíveis no passado, análises que requerem o contexto de redes completas para proporcionar resultados mais precisos e robustos em altos níveis de granularidade \cite{boyack2019creation}.

As bases de dados Scopus e Web of Science (WoS) são as duas mais usadas em estudos bibliométricos \cite{boyack2019creation}. Parte dessa popularidade se deve ao fato de serem abrangentes e cobrirem quase todas as áreas do conhecimento. Todavia, poucos pesquisadores têm acesso completo às bases de dados, uma vez que são necessárias licenças específicas para o acesso.


\section{Redes de coautoria científica}

... ... .. ..


\section{Plataforma Lattes}

A Plataforma Lattes é composta por um conjunto de bancos de dados e ferramentas construídas para auxiliar atividades de planejamento e gestão da ciência no Brasil. Armazenando informações sobre estudantes e pesquisadores do país, além de instituições e grupos de pesquisa em atividade, a Plataforma conta com uma grande variedade de fatos e informações sobre pesquisa e desenvolvimento \cite{medeiros2013dynamics}.

O Currículo Lattes, um dos bancos de dados da Plataforma, foi um meio encontrado para padronizar o histórico das atividades profissionais, cientificas e acadêmicas dos pesquisadores. A maioria dos pesquisadores brasileiros de todas as áreas do conhecimento possui seu currículo registrado na Plataforma \cite{mena2014brazilian} e o atualiza como parte da sua rotina \cite{medeiros2013dynamics}. Atualmente, grande parte das instituições acadêmicas usam esses dados para elaborar relatórios acadêmicos de produtividade e pesquisa \cite{mena2009scriptlattes}.

Assim, por ser reconhecido pela comunidade acadêmica brasileira, ter se tornado um padrão nacional para a avaliação de atividades acadêmicas e profissionais, conter um grande número de currículos de cadastrados, os quais se mantêm atualizados e por isso incluem desde a formação acadêmica até as produções acadêmicas e profissionais mais recentes dos pesquisadores, o Currículo Lattes, que atualmente consta com mais de 6 milhões de currículos, se mostra um banco de dados suficientemente abrangente para o estudo da ciência brasileira.

\chapter[Trabalhos correlatos]{Trabalhos correlatos}

Alguns trabalhos têm abordado redes de coautoria para compreender como pesquisadores dentro de sua área de conhecimento \cite{mena2014brazilian} \cite{franceschet2011collaboration} \cite{santin2016collaboration}, grupos de pesquisa \cite{delgado2014analyzing} e instituições \cite{ioannidis2008measuring} interagem.

Métodos para avaliar a contribuição de pesquisadores dentro de uma rede de coautoria \cite{liu2005co} \cite{franceschet2011collaboration}, e caracterizar a interação de pesquisadores em diferentes regiões e estados brasileiros \cite{sidone2016ciencia} também já foram propostos.

Currículos da Plataforma Lattes foram utilizados para construir redes brasileiras de coautoria que serviram para avaliar a produtividade e caracterizar o comportamento de pesquisadores brasileiros por períodos \cite{mena2014brazilian}.

Dentro da Ciência da Computação, foi possível notar que, mundialmente, pequenas listas de dois ou três autores são as mais comuns em publicações científicas, autores produtivos são os que possuem maior número de coautores, e a rede de coautoria apresenta características de redes de mundo pequeno \cite{franceschet2011collaboration}.

A ferramenta scriptLattes \cite{mena2009scriptlattes} está preparada para analisar a produção científica de um grupo de pesquisadores cadastrados na Plataforma Lattes \cite{mena2013prospecccao} e serviu como ponto de partida para a construção das redes brasileiras de coautoria mencionadas.

Segundo \citeonline{mena2014brazilian}, a rede brasileira de coautoria possui natureza interdisciplinar, com um número crescente de pesquisadores e publicações científicas feitas em coautoria.


\chapter{Conjunto de dados}

Para mensurar o impacto que a Ciência da Computação exerce nas demais áreas do conhecimento, é necessário partir de um conjunto de dados que englobe publicações científicas de todas as áreas, de tal modo que todas possam estar representadas.

Mundialmente, os bancos de dados mais usados para estudos bibliométricos são o Scopus (Elsevier) e Web of Science (WoS, Clarivate Analytics). Segundo \citeonline{boyack2019creation}, isso se deve ao fato de cobrirem grande parte das áreas de conhecimento e incluir as referências citadas, facilitando a criação de redes de citações.

No cenário brasileiro, a Plataforma Lattes, um banco de dados que armazena o currículo de diversos pesquisadores, é uma fonte de dados interessante por ter se tornado um padrão nacional para a avaliação de atividades acadêmicas e profissionais, onde a maioria dos pesquisadores de todas as áreas do conhecimento estão registrados \cite{mena2014brazilian}.

Assim, contando com mais de um milhão de currículos de pesquisadores que incluem desde sua formação acadêmica até as publicações científicas mais recentes, passando por suas áreas de atuação e premiações, a Plataforma Lattes se mostra um banco de dados adequado para o estudo.

Como o custo computacional para processar milhões de currículos e dezenas de milhões de publicações científicas seria bastante alto, decidiu-se trabalhar apenas com os currículos de pesquisadores com nível de formação acadêmica acima de doutorado.

Este conjunto de currículos foi cedido pelo Grupo de Pesquisa em Cientometria do CMCC/UFABC no formato XML, que realizou a coleta dos currículos em janeiro de 2019.
\chapter[Método]{Método}

O método utilizado é uma interpretação do processo proposto por \citeonline{fayyad1996data} para descoberta de conhecimento, onde o objetivo principal é transformar dados que são muito volumosos em formas mais compactas, abstratas ou úteis para se encontrar padrões e extrair informação.

A \autoref{fig:processo} apresenta o processo no qual, a partir dos currículos de pesquisadores, são obtidos indicadores em redes de coautoria inter- e intra-áreas que possibilitam entender o impacto que uma área do conhecimento exerce em outra.

\begin{figure}[htpb]
  \centering
  \includegraphics[scale=.3]{figuras/fayyad-diagram-graph}
  \caption{Uma visão geral dos passos adotados para o processo de descoberta de conhecimento.}
  \label{fig:processo}
\end{figure}

Este é um processo interativo e iterativo, que envolve múltiplos passos com decisões tomadas pelo pesquisador. Nas seções seguintes são discutidos os passos desse processo de descoberta de conhecimento.

\section{Seleção}

A partir de um conjunto de currículos de pesquisadores, foram extraídas informações sobre as áreas de atuação e publicações científicas, como livros, capítulos de livro, publicações em conferências e publicações em periódicos.

O \autoref{alg:selecao} apresenta o procedimento adotado, onde todos os atributos extraídos são marcados com um identificador para o currículo. Os tipos de publicações considerados neste trabalho são os livros publicados, capítulos de livro publicados, publicações completas em conferências e publicações completas em periódicos.

\begin{algorithm}
\caption{Extração de atributos do currículo do pesquisador}
\label{alg:selecao}
\begin{algorithmic}[1]

\Procedure{ExtraiaAtributos}{$id, curriculo$}
\State $A\gets \Call{SelecioneAreasDeAtuacao}{id,curriculo}$
\State $P_1\gets \Call{SelecioneLivrosPublicados}{id,curriculo}$
\State $P_2\gets \Call{SelecioneCapitulosDeLivroPublicados}{id,curriculo}$
\State $P_3\gets \Call{SelecionePublicacoesCompletasEmConferencias}{id,curriculo}$
\State $P_4\gets \Call{SelecionePublicacoesCompletasEmPeriodicos}{id,curriculo}$
\State \Return $(A,P_1,P_2,P_3,P_4)$
\EndProcedure

\end{algorithmic}
\end{algorithm}

\section{Pré-processamento}

Diversas produções científicas nessa área \cite{franceschet2011collaboration} \cite{mena2013prospecccao} \cite{reuther2006managing} descrevem casos onde uma publicação têm diversos nomes (sinônimos) e casos onde diferentes publicações possuem o mesmo nome (homônimos), cujos requerem tratamento a fim de obter um resultado mais confiável na etapa seguinte.

Neste trabalho, publicações sinônimas acontecem porque não foram cadastradas no currículo Lattes com exatamente a mesma informação, por exemplo, o mesmo título, haja vista que podem ocorrer abreviações de palavras, omissões de pontuação, ou erros de digitação.

Para amenizar este cenário foi proposto o \autoref{alg:normalizacao}, que normaliza os títulos das publicações removendo diacríticos\footnote{Um diacrítico é um sinal gráfico que se coloca sobre, sob ou através de uma letra para alterar a sua realização fonética, isto é, o seu som, ou para marcar qualquer outra característica linguística.} e elementos como tags de marcação de hipertexto, que não pertencem ao título e foram identificadas no decorrer deste trabalho.

\begin{algorithm}
\caption{Normalização do título de publicações}
\label{alg:normalizacao}
\begin{algorithmic}[1]

\Procedure{NormalizeTitulo}{titulo}
\State $titulo\gets \Call{DecodifiqueCaracteresEscapadosParaHtml}{titulo}$
\State $titulo\gets \Call{RemovaMarcacaoDeHipertexto}{titulo}$
\State $titulo\gets \Call{NormalizeParaAFormaNFKD}{titulo}$
\State $titulo\gets \Call{RemovaDiacriticosDeCaracteresASCII}{titulo}$
\State $titulo\gets \Call{NormalizeParaAFormaNFC}{titulo}$
\State \Return $titulo$
\EndProcedure

\end{algorithmic}
\end{algorithm}

Publicações homônimas são mais difíceis de ocorrer pois serão comparadas apenas publicações de um mesmo tipo referentes a um mesmo ano. Ainda, para haver o casamento entre publicações, é necessário que os títulos tenham pelo menos 3 palavras, para evitar que editoriais, por exemplo, sejam considerados uma coautoria.

Caso publicações homônimas não sejam identificadas corretamente será possível observar que a lista de coautores de um autor se dividirá em dois ou mais grupos altamente interconectados, mas sem colaborações entre grupos diferentes \cite{franceschet2011collaboration}. Ao detectar casos como este, poderá ser realizado tratamento manual.

\section{Transformação}

Usando as publicações obtidas nos currículos Lattes é possível identificar dois pesquisadores como coautores se uma mesma publicação aparece em seus currículos, e com isso construir uma rede de coautoria onde os nós são pesquisadores e os vértices a coautoria entre eles. O mesmo procedimento pode ser feito por períodos (\textit{e.g.}, biênio ou triênio), onde se obtém uma série de redes de coautoria com espaçamento temporal.

Para identificar as coautorias entre quaisquer dois autores foi proposto o \autoref{alg:coautoria}, que compara todas as publicações dentro de um mesmo ano usando o algoritmo de distância Levenshtein (\autoref{alg:levenshtein}).

\begin{algorithm}
\caption{Identificação de coautorias}
\label{alg:coautoria}
\begin{algorithmic}[1]

\Procedure{IdentificaCoautoria}{P}
\State $P^2\gets P\times{P}$
\State $C\gets \emptyset$
\ForAll{$(p_1, p_2) \in P^2$}
\If{publicações já foram comparadas}
\State \textbf{continue}
\ElsIf{publicações de anos diferentes}
\State \textbf{continue}
\ElsIf{publicações do mesmo pesquisador}
\State \textbf{continue}
\ElsIf{títulos compostos por duas palavras ou menos}
\State \textbf{continue}
\ElsIf{títulos com diferença superior a 85\% no tamanho}
\State \textbf{continue}
\ElsIf{\Call{Levenshthen}{\Call{Titulo}{$p_1$},\Call{Titulo}{$p_2$}} $<$ 0.85}
\State \textbf{continue}
\Else
\State $C\gets C\cup\{(p1, p2)\}$
\EndIf
\EndFor

\State \Return $C$
\EndProcedure

\end{algorithmic}
\end{algorithm}

\begin{algorithm}
\caption{Identificação de coautorias}
\label{alg:levenshtein}
\begin{algorithmic}[1]

\Procedure{Levenshtein}{s}
\State \Return $i$
\EndProcedure

\end{algorithmic}
\end{algorithm}

\section{Mineração de dados}

Nesta etapa serão feitos os cálculos de indicadores das redes de coautoria, onde planeja-se considerar as seguintes métricas topológicas: distância média, centralidade de grau, centralidade de proximidade, centralidade de autovetor, centralidade de contribuição, PageRank e AuthorRank\footnote{O AuthorRank é definido como uma medida que descreve as interações de um autor na rede de coautoria \cite{liu2005co}.}.

Pretendemos ainda, explorar novos arcabouços computacionais para processar todos os dados coletados.

\section{Identificação de padrões e avaliação}

Será determinada a existência de alguma correlação entre os indicadores obtidos e a análise desses indicadores para determinar se existe influência (ou impacto relativo) da Ciência da Computação em outras áreas.

Complementarmente, acontecimentos históricos que possam justificar o padrão encontrado serão buscados a fim de contextualizar a avaliação.

Diferentes conceitos de aprendizado de máquina e reconhecimento estatístico de padrões serão abordados.

\chapter{Resultados e Discussão}
Nesta seção apresentamos ...

\section{Produção bibliográfica}

A produção bibliográfica da Ciência da Computação está posicionada entre as 10 primeiras.

Se olharmos pelo tipo de publicação com maior volume vemos que o número de publicações em conferências é alto. Ainda, esse é um comportamento mais presente na Engenharia elétrica e etc.

Podemos notar um aumento no número de publicações bibliográficas, mas a taxa de crescimento segue o que é visto nas demais áreas, e portanto não notamos grandes mudanças após data X, quando a ciencia da computação...

\section{Colaboração}

A colaboração da ciencia da computação é expressiva apenas em algumas áreas. Ela está ligada à maioria das áreas, porém esse é um comportamento comum pelo que se pôde observar.

Ao longo do tempo podemos notar

\section{Interdisciplinaridade}

Interdisciplinaridade é um conceito que pode ser interpretado de diferentes maneiras. Vemos que a ciencia da computação participa com autores das mais variadas áreas, porém pela pouco impacto sofrido de outras áreas e pouco impacto exercido em outras áreas, com exceção da engenharia elétrica, engenharia de biomedicina e tal, não podemos afirmar que esta é uma área interdisciplinar.

Diferente da educação, que tem uma grande participação em boa parte das áreas. Ainda, vale notar que este é um resultado esperado e que pode servir como um indicador de que o método aplicado nos dados é válido e produz resultados que correspondem à realidade.


\chapter{Considerações finais}

Foram construídas as redes de coautoria dos pesquisadores com nível igual ou acima de doutorado cadastrados na Plataforma Lattes. O objetivo era caracterizar o impacto que a Ciência da Computação exerce nas demais áreas do conhecimento para responder quais são as áreas que mais interagem com a Ciência da Computação, mensurar o impacto nas demais áreas do conhecimento, bem como o comportamento dessa medida ao longo do tempo.

Os resultados obtidos apontam que a Ciência da Computação é uma das cinco áreas que mais produzem publicações científicas em coautoria e que esse número vem crescendo consistentemente ao longo dos anos. Todavia, com relação à participação em grupos interdisciplinares para a produção de publicações científicas, a Ciência da Computação não está bem posicionada. Áreas como a Química e a Agronomia, que apresentam desempenho similar na produção de publicações em coautoria, também estão entre as primeiras colocadas neste quesito.

Apesar de ser uma área que se conecta a quase todas as áreas do conhecimento, as coautorias realizadas por cientistas da computação não exercem impacto significativo na produção das publicações científicas de outras áreas. Assim, mesmo observando que o uso da computação vem sendo cada vez mais explorado, isso não se traduz em um pesquisador da área participando do processo, o que abre espaço para novas discussões e sugere que pesquisadores de outras áreas estão se capacitando para o uso da computação como ferramenta.

\section{Limitações}

A base de dados é um retrato de janeiro de 2019 e, como a obtenção dos currículos é demorada, não é possível que o processo de descoberta do conhecimento considere sempre a versão mais atualizada dos dados. Como a identificação das coautorias dura aproximadamente 12 dias em uma máquina com dois núcleos de processamento, mesmo que um novo retrato da base de dados pudesse ser obtido a cada hora, não seria possível trabalhar com a versão mais recente do conjunto de dados.

O tempo de processamento é alto porque o algoritmo usado para a identificação das coautorias, mesmo com as otimizações, continua sendo $O(n^2)$ para uma lista de publicações de tamanho $n$. Ou seja, o tempo de execução aumenta quadraticamente em relação ao tamanho da entrada.

As análises feitas abordam o retrato atual das coautorias, mas não são capazes de informar se existiram mudanças nas relações de coautoria entre as áreas e como esse comportamento se deu ao longo dos anos. Ainda, nem todos os pesquisadores preenchem suas áreas e isso faz com que tenhamos que desconsiderar parte dos dados. Inclusive existem coautorias que foram desconsideradas devido a esse tipo de dado faltante.

\section{Trabalhos futuros}

Existem muitas análises que podem ser derivadas das redes de coautoria construídas, uma vez que elas dão o panorama geral das coautorias de todos os pesquisadores com nível de instrução igual ou acima de doutorado, os quais são os mais relevantes na produção científica brasileira.

Os dados presentes no Currículo Lattes permitem entender como funcionam as coautorias entre pesquisadores de universidades e empresas, de diferentes regiões do país e do mundo, além do impacto que falar uma língua estrangeira pode ter no aumento das coautorias

Também poderiam ser exploradas propriedades dessas redes, como a identificação de pesquisadores produtivos e influentes, além de observar a hipótese de que os autores mais produtivos são os que mais colaboram.

Momentos históricos do país poderiam ser avaliados contra o volume de publicações feitas em coautoria. A \autoref{fig:coautoriaanual} apresenta de 2012 para 2013 uma desaceleração no crescimento de publicações feitas em coautoria, o que coincide com a greve das universidades federais. Seria relevante descobrir se manifestações sociais e políticas, assim como surtos de doenças, influenciam as relações da ciência.

Ainda, poderia ser construída uma ferramenta para desambiguação autores, dado que identificar todos os coautores de uma publicação em citações é problema. Esse problema é ainda pior quando apenas o nome de um dos pesquisadores é colocado nas citações.

% \include{capitulos/limitacoes}
% \include{capitulos/trabalhosfuturos}

% ----------------------------------------------------------
% ELEMENTOS PÓS-TEXTUAIS (Referências, Glossário, Apêndices)
% ----------------------------------------------------------
\postextual

% Referências bibliográficas
\bibliography{bibliografia}

% Glossário (Consulte o manual)
%\glossary

% Apêndices
% ----------------------------------------------------------
% Apêndices
% ----------------------------------------------------------

% ---
% Inicia os apêndices
% ---
\begin{apendicesenv}

% Imprime uma página indicando o início dos apêndices
\partapendices

% ----------------------------------------------------------
\chapter{Diagrama da estrutura do currículos Lattes}
\label{ap:diagramacurriculo}
% ----------------------------------------------------------

\begin{figure}[htpb]
  \centering
  \includegraphics[width=1\textwidth]{figuras/diagrama-curriculo-dados-gerais}
  \caption{Uma visão geral dos nós do currículo Lattes relativos aos dados do pesquisador.}
  \label{fig:diagramadadosgerais}
\end{figure}

\begin{figure}[htpb]
  \centering
  \includegraphics[width=1\textwidth]{figuras/diagrama-curriculo-producao-bibliografica}
  \caption{Uma visão geral dos nós do currículo Lattes relativos à produção bibliográfica.}
  \label{fig:diagramaproducaobibliografica}
\end{figure}

% ----------------------------------------------------------
\chapter{Diagrama entidade-relacionamento do banco de dados completo}
% ----------------------------------------------------------

% ----------------------------------------------------------
\chapter{Exemplos de coautorias identificadas na etapa de transformação do método}
% ----------------------------------------------------------

% ----------------------------------------------------------
\chapter{Exemplos de áreas de atuação do pesquisador inferidas na etapa de transformação do método}
% ----------------------------------------------------------

\end{apendicesenv}
% ---

% Anexos
% \include{postextual/anexos}

% Índice remissivo (Consultar manual)
%\phantompart
%\printindex

\end{document}
